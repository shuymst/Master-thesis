\documentclass[11pt, dvipdfmx, openany]{jsbook}
\usepackage[dvipdfmx]{graphicx}
\usepackage{hyperref}
\usepackage{pxjahyper}
\usepackage[a4paper]{geometry}
\usepackage{amsmath,amssymb}
\usepackage{amsfonts}
\usepackage{algorithm}
\usepackage[noend]{algpseudocode}
\usepackage{bm}
\usepackage{here}
\usepackage{color}
\usepackage[hang,small,bf]{caption}
\usepackage[subrefformat=parens]{subcaption}
\hypersetup{colorlinks=false, hidelinks}
\newcommand{\red}[1]{\textcolor{red}{#1}}
\renewcommand{\baselinestretch}{1.1}
\captionsetup{compatibility=false}
\usepackage{url}

\DeclareMathOperator*{\argmax}{arg\,max}
\DeclareMathOperator*{\argmin}{arg\,min}

\setlength{\textwidth}{\fullwidth}  %本文の幅(textwidth)を全体の幅(=ヘッダ部の幅)にそろえる
\setlength{\evensidemargin}{\oddsidemargin} %偶数ページの余白と奇数ページの余白をそろえる

\begin{document}
\title{確率的ゲーム2048の強化学習の研究}
\author{山下修平}
\maketitle 

\tableofcontents
\clearpage

% 本文
\chapter{はじめに}
強化学習はエージェントが環境での試行錯誤を繰り返すことで, 賢い行動を学習するための枠組みである.
強化学習は機械の制御や広告の最適化など広い適用範囲を持つが, 中でもゲームAIは最も強化学習と関わりの深い分野の$1$つである.
ゲームを賢くプレイすることは, 様々な能力を必要とするため, 強化学習のベンチマークとして用いられてきた.
近年では深層学習・深層強化学習の技術の発展により, ゲームAIは大きな進歩を遂げている.
Deep Q Network~(DQN)~\cite{DQN}がAtariのいくつかのゲームで人間を超えたことは有名である.
またAlphaGo~\cite{AlphaGo}は囲碁のトッププロを破ったことで世間の大きな注目を集めた.
一方で囲碁や将棋に代表される二人零和有限完全確定情報ゲームや研究の盛んなAtariは, いずれも環境のダイナミクスが決定的なゲームであり, ランダム性は介入しない.
2048は確率的ゲームである.
\chapter{2048}

\section{2048のルールと用語説明}
\label{sec:rule}
2048は, Gabrirle Cirulliによって公開された$1$人用のパズルゲームである~\cite{2048}.
ゲームは$16$マスからランダムに選ばれた$2$マスに$2$か$4$の数字タイルが置かれた盤面から始まる.
プレイヤが行うことは上下左右いずれかの方向を選択することである. 
プレイヤがある$1$つの方向を選ぶと, 盤面上のすべての数字タイルは選択した方向に向かってスライドして移動する.
スライドする数字タイルは空きマスを通過し, 異なる数字タイルの直前か盤面の端で停止する.
スライドして移動する際に$2$つの同じ数字のタイルが衝突すると, これらは合体してその合計の数字の$1$つのタイルへ変化し, プレイヤはその数値を得点として獲得する.
そのため, ゲームには$2$の累乗の数字タイルしか現れない.
図~\ref{fig:all_directions}にある盤面から上下左右を選択したときの, 数字タイルのスライドの仕方の具体例を示す.
\begin{figure}[t]
    \centering
    \includegraphics[width=0.8\linewidth{}]{figures/all_directions.pdf}
    \caption{上下左右それぞれへのスライドの例 \label{fig:all_directions}}
\end{figure}

数字タイルのスライド後, 空きマスから等確率に選択されたある1マスに$90\%$の確率で$2$のタイルが, $10\%$の確率で$4$のタイルが置かれる. 
ゲームはプレイヤの行動による数字タイルのスライドと新たな数字タイルの出現を交互に繰り返して進行する.
盤面上の数字タイルが市松模様のようになると, プレイヤが選択可能な行動がなくなったときにゲームは終了する~(図~\ref{fig:terminal}を参照).
\begin{figure}[t]
    \centering
    \includegraphics[width=0.2\linewidth{}]{figures/terminal_.pdf}
    \caption{終了状態の例 \label{fig:terminal}}
\end{figure}

ここでプレイヤが行動を選択する盤面を\textgt{状態}, 行動を選択して新たな数字タイルが出現する直前の盤面を\textit{afterstate}と呼ぶ.
図~\ref{fig:transition}に状態$s$からafterstate $s'$を経由して, 次の状態$s_{\text{next}}$に遷移する例を示す.

\begin{figure}[t]
    \centering
    \includegraphics[width=\linewidth{}]{figures/transition_.pdf}
    \caption{状態遷移の例 \label{fig:transition}}
\end{figure}

プレイヤの一般的な目標はゲームのタイトルが示す$2^{11}=2048$のタイルを完成させることだが, それ以降もゲームを続けることができる.

\section{ゲームの進行と時刻}
\label{sec:property}
2048はゲームの性質上, 状態からafterstateへの遷移において盤面上の数字タイルの合計値は不変である~(図~\ref{fig:all_directions}を参照).
盤面上の数字タイルの合計値はafterstateから次の状態への遷移においてのみ変化する.
新しい数字タイルとして$2$か$4$のタイルが出現することで, 数字タイルの合計値はその値の分だけ必ず増加する.
すなわちプレイヤが$1$回行動するたびに, 盤面上の数字タイルの合計値は$2$か$4$ずつ単調に増加する.

よって盤面上の数字タイルの合計値をゲームの進行度合いとして用いることができる.
以降これを\textgt{時刻}と呼ぶ.
例えば図~\ref{fig:transition}では時刻$2\times4+4+8\times3+16=52$の状態$s$が時刻$52+2=54$の状態$s_{\text{next}}$に遷移している.
ゲームの時刻はプレイヤが行動するたびに必ず増加するため, 2048はサイクルの出現しないゲームであることがわかる.



\subsection{ゲームの終了状態}

またゲームの開始盤面をまとめて初期状態と呼ぶことにする.
\chapter{2048と強化学習}

\section{強化学習の概要}

\section{2048に対する強化学習の先行研究}
\chapter{提案手法}
\label{chap:proposal}
\ref{sec:rlto2048}節で述べたように, 2048を対象とした強化学習の研究は数多くなされてきた.
もしゲームが完全に解かれていれば, 強化学習手法の良し悪しを定量的に評価することができる.
一方で2048はゲーム木の大きさから, 完全解析を実行することは計算資源の観点から困難である.
そこで本研究では2048のミニゲームの完全解析を行うことを提案する.
さらに解析したミニゲームをベンチマークとして, 2048の強化学習手法について詳細に検討する.

\section{2048のミニゲームの完全解析}
\label{chap:solving}
\ref{chap:rl}章で述べた強化学習は環境~(ゲーム)~と何度もやり取りすることで, 最適な方策を学習するための手法である.
一方で小さなゲームであれば, 力ずくの計算によってゲームを完全に解くこともできる.
本節では2048を解析的なアプローチによって解くことについて述べる.
なお本節の内容は文献~\cite{3x3_2048}および文献~\cite{4x3_2048}を元に執筆された.

\subsection{2048の完全解析とは}
\label{sec:solving}
2048は$1$人用のゲームであるため, 勝敗のようなプレイヤの明確な目標は存在しない.
そのためプレイヤが何を目標とするかによって, プレイヤの最善手の定義は変化する.
また\ref{sec:rule}節で述べたようにゲームはランダム性を伴うため, 同じ状態から毎回同じ手を選んでも結果は確率的に変動する.

そこで本稿ではある状態$s$における最善手を「$s$から獲得できる得点の合計の期待値が最も高くなるような手」と定義する.
これは~\ref{chap:rl}節で述べた強化学習の最適状態価値と等価なものである.
よって状態$s$から最善手を選び続けて獲得できる得点の合計の期待値を状態$s$の最適価値と呼び, $v_*(s)$で表すことにする.

このとき$v_*(s)$は式~\ref{eq:value}のように再帰的な形式で書くことができる.
\begin{align}
    v_*(s) =
    \begin{cases}
        0 & (s \text{が終了状態}) \\
        \max_a \left(r(s,a) + \mathbb{E}_{s_\text{next} \in \mathcal{T}(s,a)} v_*(s_\text{next}) \right) & (\text{otherwise})
    \end{cases}
    \label{eq:value}
\end{align}
ただし$r(s,a)$は状態$s$から行動$a$をとって獲得する得点, $s_\text{next} \in \mathcal{T}(s,a)$は状態$s$から行動$a$をとって遷移しうる次の状態の集合を表す~(図~\ref{fig:state_afterstate}を参照).
式~\ref{eq:value}の$r(s,a) + \mathbb{E}_{s_\text{next} \in \mathcal{T}(s,a)} v_*(s_\text{next})$は, 強化学習における最適行動価値$q_*(s,a)$に対応する.
また$\mathbb{E}_{s_\text{next} \in \mathcal{T}(s,a)} v_*(s_\text{next})$は, $s$から$a$をとって遷移するafterstate $s'$の価値といえる.

\begin{figure}[t]
    \centering
    \includegraphics[width=0.6\linewidth{}]{figures/value_function_.pdf}
    \caption{式~\ref{eq:value}の補足図}
    \label{fig:state_afterstate}
\end{figure}

ゲームに現れうるすべての状態の最適価値を計算すれば, 任意の状態において最善手を選ぶことができる.
本稿ではこれを2048の完全解析ということにする.

完全解析をすることで, ゲームの任意の状態の最適価値・最善手を明かし, 最善手を選び続けるプレイヤの戦略を解析することができる.
さらに2048を対象とした強化学習手法の良し悪しを, 定量的な指標によって評価することができると考えられる.
一方で2048を完全解析することは, そのゲーム木の大きさによる計算コストの観点から現状難しいと考えられる.
そこで本研究では本来$4\times4$盤面上で行われる2048のミニゲームとして, 盤面サイズを縮小した2048を完全解析することを提案する.

\subsection{盤面サイズが小さな2048の完全解析}
\label{sec:mini2048}
基本的なルールは2048と同じで盤面サイズを$4\times4$から縮小したゲームを完全解析することを考える.
盤面サイズに関わらず, 以下の$2$つのステップを順番に行うことで完全解析を実行することができる.
\begin{enumerate}
    \item 初期状態から到達し得るすべての状態の列挙
    \item 列挙した状態の最適価値の計算
\end{enumerate}

\subsubsection{初期状態から到達し得るすべての状態の列挙}
\label{subsec:enumeration}
完全解析の第1ステップとしてゲームに現れうるすべての状態を列挙する.
これまでに発見した状態の集合$S$から状態を$1$つ取り出し, $s$から遷移可能な次の状態$s_{\text{next}} \in \mathcal{T}(s)$の内, 未発見の状態を$S$に追加する.
初期状態を$S$に入れて列挙を開始し, 新たに発見する状態がなくなるまで繰り返すことで, すべての状態を列挙することができる.

素朴な方法ではこれまでに発見した状態の集合$S$をメモリ上で管理することが考えられるが, 状態数が非常に大きな場合にはメモリの容量を超えてしまう.
そこで~\ref{sec:property}節で説明した時刻によって, ゲーム木を整理しこれを解決する.
時刻$t$の状態は時刻$t+2$か$t+4$の状態にしか遷移しないため, 時刻$t+2$と$t+4$の発見した状態の集合$S_{t+2}$と$S_{t+4}$をメモリ上で管理すれば十分である.
よって時刻が最小の$4$の状態から時刻$2$刻みで順番に列挙を行うことで, ディスクを効率的に活用することができる.
以上を踏まえた疑似コードをAlgorithm~\ref{alg:bfs}に示す.

\begin{algorithm}[tb]
\caption{すべての状態の列挙}
\label{alg:bfs}
\begin{algorithmic}[1]
\Function {enumeration}{}
    \State INITIALIZE($S_4, S_6, S_8$)
    \For {$t=4$ to $t_{\text{max}}$}
        \ForAll {$s_t \in S_t$} 
            \ForAll {$s_{t+2} \in \mathcal{T}(s_t)$}
                \If {$s_{t+2} \notin S_{t+2}$} 
                    \State $S_{t+2} = S_{t+2} \cup \{s_{t+2}\}$
                \EndIf
            \EndFor
            \ForAll {$s_{t+4} \in \mathcal{T}(s_t)$}
                \If {$s_{t+4} \notin S_{t+4}$} 
                    \State $S_{t+4} = S_{t+4} \cup \{s_{t+4}\}$
                \EndIf
            \EndFor
        \EndFor
    \EndFor
\EndFunction
\end{algorithmic}
\end{algorithm}

\subsubsection{後退解析による状態の最適価値の計算}
\label{subsec:calculation}
\ref{subsec:enumeration}節で列挙した状態の価値を, 式~\ref{eq:value}に従って後退解析を行い計算する.
状態列挙のときと同様に, 時刻に従って状態を管理することで効率的に後退解析を行える.
すなわち時刻$t$の状態の価値は, 時刻$t+2$と$t+4$の状態の価値が計算済みであれば必ず計算できる.
よって時刻が最大の状態から順番に走査することで, 無駄なくすべての状態の価値を計算できる.
疑似コードをAlgorithm~\ref{alg:calculation}に示す.
\begin{algorithm}[tb]
    \begin{algorithmic}[1]
    \Function {calculation}{$t$}
        \For {$t=t_{\text{max}}$ to $4$}
            \ForAll {$s_t \in S_t$} 
                \If {$s_t$ is gameover}
                    \State $v_*(s_t) = 0$
                \Else {}
                    \State $v_*(s_t) = \max_a \left(r(s_t,a) + \mathbb{E}_{s_\text{next} \in \mathcal{T}(s_t,a)} v_*(s_\text{next}) \right)$
                \EndIf
            \EndFor
        \EndFor
    \EndFunction
    \end{algorithmic}
    \caption{後退解析による価値計算}
    \label{alg:calculation}
\end{algorithm}

\subsection{実験結果}
本研究では$2\times2$盤面から$4\times3$盤面までの2048を完全解析することに成功した.
完全解析した結果を表~\ref{table: analysis_table}に示す.
\begin{table}[t]
    \centering
    \begin{tabular}{rrr}
        \hline \hline
        盤面サイズ & 状態数 & 初期状態の価値\\ \hline
        $2\times2=4$ & $110$ & $67.6$ \\
        $3\times2=6$ & $21,752$ & $480.9$ \\
        $4\times2=8$ & $4,980,767$ & $2,642.6$ \\
        $3\times3=9$ & $48,713,519$ & $5,468.4$ \\
        $4\times3=12$ & $1,152,817,492,752$ & $50,724.2$ \\
        \hline
    \end{tabular}
    \caption{盤面の大きさと解析結果に関する表}
    \label{table: analysis_table}
\end{table}
盤面サイズが大きくなるに従って指数関数的に状態数は大きくなることが分かる.
そのため$4\times4$盤面の2048は完全解析を行うには状態数が非常に大きいことが予想される.
一方で初期状態の最適価値は, 盤面サイズが$n$マス増える度に$2^n \sim 2^{n+1}$倍になっていることが見て取れる.
よって$4\times4$盤面の2048では初期状態の最適価値は, 少なく見積もっても$800,000$点程度はあるのではないかと推測される.

$3\times3$盤面と$4\times3$盤面の2048の各時刻における状態数のグラフを図~\ref{fig:state_afterstate}に示す.
多くのゲームでは進行に従って多様な盤面が存在するため状態数は大きくなり続ける.
一方で2048は盤面が数字タイルで埋まるとゲームオーバーになりやすく, その時刻の状態数は少なくなる.
大きな数字タイルを完成させると盤面上に空きマスが増え, ゲームは再び複雑性を増す.
実際, 図~\ref{fig:state_afterstate}からは$2^n$の前後の時刻で状態数が大きく増減していることが見て取れる.
よって2048はゲームの進行に従って, ゲームの複雑性が増減するという特徴を持つといえる.
\begin{figure} 
\vspace{0.2cm}
\begin{subfigure}[T]{0.4\columnwidth}
    \centering
    \includegraphics[width=\columnwidth]{figures/graph_mini.pdf}
    \caption{$3\times3$盤面の2048}
    \label{fig:graph_mini}
\end{subfigure}
\hspace{1cm}
\begin{subfigure}[T]{0.4\columnwidth}
    \centering
    \includegraphics[width=\columnwidth]{figures/graph_mid.pdf}
    \caption{$4\times3$盤面の2048}
    \label{fig:graph_mid}
\end{subfigure}
\label{fig:time_state_num}
\end{figure}

参考として, $2\times2$盤面の2048のゲーム木全体を図~\ref{fig:game_tree}に示す.
ゲーム木が拡大と縮小を繰り返す様子が見て取れる.
\begin{figure}[t]
    \centering
    \includegraphics[width=0.7\linewidth{}]{figures/tree.pdf}
    \caption{$2\times2$盤面の2048のゲーム木~(赤色のノードは初期状態, 青色のノードは終了状態)}
    \label{fig:game_tree}
\end{figure}

\section{2048のミニゲームの完全解析と強化学習}
\ref{sec:solving}節では強化学習のベンチマークとしての2048のミニゲームの提案, およびその完全解析を行った.
本研究では~\ref{subsec: stochastic_muzero}節で述べた, Stochastic MuZero~\cite{StochasticMuZero}について詳細に研究する.
ただしStochastic MuZeroのように, 環境のダイナミクスモデルを学習するには多くの計算資源を要する.
そこでAlphaZero~\cite{AlphaZero}のように環境のダイナミクスは既知として, 方策・価値ネットワークの訓練のみを行う手法を考える.
これを本稿では2048-AlphaZeroと呼ぶことにする.

本節では$3\times3$盤面の2048および$4\times3$盤面の2048

\begin{figure}[t]
    \centering
    \includegraphics[width=0.7\linewidth{}]{figures/alphazero_3x3.pdf}
    \caption{$2\times2$盤面の2048のゲーム木~(赤色のノードは初期状態, 青色のノードは終了状態)}
    \label{fig:alphazero_3x3}
\end{figure}
% あと2, 3年費やせばできそうなテーマを書く
\chapter{まとめと今後の展望}
一方で2048を対象とした強化学習の先行研究は, 
内発的報酬を利用した探索の促進が考えられる~\cite{RND}.

\chapter*{謝辞}
学部$4$年時の卒業研究よりご指導いただいた金子知適先生をはじめとして, 金子研究室やゲームプログラミングセミナーの皆様に感謝申し上げます.
また修士課程への進学を支えてくれた両親に感謝します.

\bibliography{ref} % 研究に役に立ちそうならなんでも入れとく
\bibliographystyle{junsrt}

\appendix
\chapter{実装の詳細}

\section{ゲーム環境の実装}
\label{sec:game_impl}
2048は状態からafterstateへの遷移において, 各行~(列)~の変化は独立に考えることができる.
また回転と反転を考慮することで上下左右は等価な盤面変化を起こす.
よって$1$行の全パターンについて, ある一方向を選択したときの遷移先を前もって計算することで, 全方向に対する盤面全体の遷移を高速に行える.

\section{完全解析の実装}
図~\ref{fig:symmtric_boards}に示すように, 回転・反転に関して同じ盤面は$1$つの状態として扱った.
またそれぞれの状態は$64$ビット整数で表現された.
\begin{figure}[t]
    \centering
    \includegraphics[width=0.6\linewidth{}]{figures/symmetric.pdf}
    \caption{$3\times3$盤面の2048の対称盤面}
    \label{fig:symmtric_boards}
\end{figure}
完全解析にはメモリ$256$GBでプロセッサはコア数$32$のAMD Ryzen Threadripper 3970Xのマシンを用いた.
プログラムはすべてC++言語で実装された.

$4\times3$盤面の2048の完全解析は状態列挙に約$45$日, 価値計算に約$20$日を要した.
ただし実装の細かい工夫の仕方で改善することができると考えられる.
また列挙した状態とその価値の情報を保持するために合計で$16.8$TBの記憶領域を必要とした.
本研究では状態を表す$64$ビットと, 価値を表す$64$ビットを素朴に並べることでデータを記録した.
簡潔データ構造の工夫を導入し, データを圧縮することは今後の課題としたい.

\section{強化学習の実装}
\subsection{ニューラルネットワークの詳細}
\label{subsec:nn_impl}
ニューラルネットワークは盤面の特徴量を入力として, 方策と価値を出力する.
盤面サイズ$H \times W$のルールの下では理論上の最高到達タイルは$2^{H \times W + 1}$である.
このとき入力は空きマス$H \times W + 2$チャネルの$H \times W$から成る.
$n$番目のチャネルの$(i,j)$成分には盤面の$(i,j)$
\begin{figure}[t]
    \centering
    \includegraphics[width=0.6\linewidth{}]{figures/encoding.pdf}
    \caption{ニューラルネットワークへの入力特徴量}
    \label{fig:input_encoding}
\end{figure}

またStochastic MuZero~\cite{StochasticMuZero}に倣って, ニューラルネットワークは価値を$D$次元のベクトル値として出力する.
価値の学習ターゲット$x$~(スカラー値)~は, まず$h(x)= \sqrt{x+1} - 1 + \epsilon x \ (\epsilon=0.001)$によってスケールを調整する~(図~\ref{fig:transform}を参照).
\begin{figure}[t]
    \centering
    \includegraphics[width=0.6\linewidth{}]{figures/transform_.pdf}
    \caption{$h(x)$のグラフ}
    \label{fig:transform}
\end{figure}
さらに$h(x)$はtwo-hotという, 特定の$2$つの要素以外は全ての$0$であるようなベクトル値に変換される.
たとえば$h(x)=3.7$の場合, $3.7=3 \times 0.3 + 4 \times 0.7$であるため, $3$の重みが$0.3$, $4$の重みが$0.7$, それ以外は$0$である$D$次元ベクトルに変換される.
これをニューラルネットワークの価値の学習ターゲットとして, Cross Entropy誤差を最小化するように訓練される.
推論時にはニューラルネットワークの価値の出力を, softmaxにより全体の総和を$1$にする.
これを$0$から$D$までのそれぞれの重みとして, 重み付け平均$y$を計算する.
最後に$h(x)$の逆関数である$h^{-1}(y)= \epsilon^{-1} (y + 0.5\epsilon^{-1} + 1.0) - 0.5 \sqrt{4.0 \epsilon^{-3} y + 1.004 \epsilon^{-4}}$によって, $x=h^{-1}(y)$を得る.
$3\times3$盤面の2048の実験では$D=200$, $4\times3$盤面では$D=400$, $4\times4$盤面では$D=600$とした.

\subsection{Prioritized Experience Replayの実装}
Prioritized Experience Replay~\cite{prioritized}は学習に使用するデータを一様ランダムではなく, priorityと呼ばれる重みに従ってサンプルするExperience Replayである.
$i$番目のデータのpriorityを$p_i$とする.
このとき$i$番目のデータはハイパーパラメータ$\alpha$を用いて, 確率$P(i) = \frac{p_{i}^{\alpha}}{\Sigma_k p_{k}^{\alpha}}$でサンプルされる.

Prioritized Experience Replayからのデータのサンプル, およびpriorityの更新は, sum-treeという二分木でデータを管理することで$\mathcal{O}(\log n)$で行うことができる.
sum-treeの葉ノードはpriorityの値を保持し, 各ノードは左右の子ノードの合計値を保持する.
そのため根ノードはpriority全体の合計値$S$を持つ.
サンプルする際には, まず$0$から$S$までの値をランダムに生成し, 根ノードから葉ノードに至るまでたどることで選ぶ.
図~\ref{fig:sumtree}にsum-treeと具体的なサンプルの仕方を例示する.
\begin{figure}[t]
    \centering
    \includegraphics[width=0.4\linewidth{}]{figures/sumtree_.pdf}
    \caption{sum treeの例}
    \label{fig:sumtree}
\end{figure}

\chapter{2048のゲーム性とプレイヤの戦略の検証}
ここでは$3\times3$盤面の2048を主な題材として, 2048のゲーム性とプレイヤの戦略について様々な面から検証する.

\section{2と4の出現確率とプレイヤの得点}
\ref{sec:rule}節で述べたように, 2048の通常のルールではafterstateが次の状態へ遷移する際に出現する, 新しいタイルの数字と位置はランダムに決まる.
すなわちafterstateの空きマスから等確率に選択されたある1マスに$90\%$の確率で$2$のタイルが, $10\%$の確率で$4$のタイルが置かれる.

ここで$2$のタイルと$4$のタイルの出現確率を変更した場合にゲーム性がどう変わるか検証する.
表~\ref{table: value_table}に$3 \times 3$盤面の2048において, $4$の出現確率を通常の$10\%$から増減させたときの完全解析の結果を示す.
表から分かるように, $2$と$4$の出現確率がいずれか一方に傾くほど期待値は大きくなる傾向があることが分かる.
また$4$の出現確率が$0\%$の場合と$100\%$の場合には, 最適な行動をし続ければ常に~図\ref{fig:limit}のような理論上の最終盤面に到達できることが分かる.
2048は$2$と$4$の$2$種類の数字タイルが良い割合で出現することが, 人間がプレイするにあたってゲームを面白くしていると考えられる.
\begin{table}[t]
\caption{4の出現確率を増減させたときのゲームの期待値}
\label{table: value_table}
\centering
\begin{tabular}{r|r||r|r}
    \hline \hline
    4の確率 & 初期状態の期待値 & 4の確率 & 初期状態の期待値 \\ \hline 
    0.00 & 7172.00 & 0.55 & 3206.00 \\
    0.05 & 6161.17 & 0.60 & 3171.24 \\
    \textbf{0.10} & \textbf{5468.49} & 0.65 & 3165.36 \\
    0.15 & 4932.54 & 0.70 & 3194.44 \\
    0.20 & 4515.42 & 0.75 & 3269.18 \\
    0.25 & 4182.44 & 0.80 & 3399.20 \\
    0.30 & 3919.20 & 0.85 & 3607.78 \\
    0.35 & 3704.44 & 0.90 & 3993.30 \\
    0.40 & 3531.46 & 0.95 & 4938.20 \\
    0.45 & 3390.19 & 1.00 & 14344.00 \\
    0.50 & 3278.70 &  & \\
    \hline
\end{tabular}
\end{table}

\section{2048とminmax法}
環境が常にプレイヤにとって最も都合の悪くなるように, 新しい数字タイルを出現させるゲームを考える.
これは二人零和有限完全確定情報ゲームと同様に, minmax法によって完全解析を行える.
よって本稿ではこの場合の環境をminmax環境と呼ぶことにする.
minmax環境の2048は, 得点を最大化したいプレイヤと得点を最小化したい環境の対戦ゲームのように考えることができるため, 「対戦型2048」とも呼ばれている~\cite{battle_2048}.
minmax環境における状態の価値$v_{\text{minmax}}$は, 以下の式~\ref{eq:minmax}に従って, ~\ref{sec:solving}節と同様に後退解析を行い計算できる.
\begin{align}
    v_{\text{minmax}}(s) =
    \begin{cases}
        0 & (s \text{が終了状態}) \\
        \max_a \left(r(s,a) + \min_{s_\text{next} \in \mathcal{T}(s,a)} v_{\text{minmax}}(s_\text{next}) \right) & (\text{otherwise})
    \end{cases}
    \label{eq:minmax}
\end{align}

$3\times3$盤面のminmax環境におけるプレイヤと環境の最善手順を図~\ref{fig:minmax_env}に示す.
ゲームは$21$手で終了し, プレイヤは$164$点しか獲得することができない.
環境がプレイヤの妨害をすると, プレイヤの得点は最善手を選んだとしてもごく僅かになることがわかる.
そのため対戦型2048において環境がプレイヤにとって最悪手を選択する場合, 通常の2048と比べてプレイヤ視点でのゲーム性に欠けるだろう.
対戦型2048に良いゲーム性を持たせるには, 環境側に何らかの制約を加えるなどの工夫をする必要があると考えられる.
\begin{figure}[t]
    \centering
    \includegraphics[width=\linewidth{}]{figures/minmax_transition.pdf}
    \caption{$3\times3$盤面のminmax環境における双方の最善手順}
    \label{fig:minmax_env}
\end{figure}

minmax環境と同様に, 環境がプレイヤの得点を最大化させるように振る舞うようなゲームを考えることもできる.
これは式~\ref{eq:minmax}中の$\min$を$\max$に置き換えることで完全解析を行える.
よってこの場合の環境をmaxmax環境と呼ぶことにする.
直感的にも明らかなように, maxmax環境では必ず図~\ref{fig:limit}に示したような理論上の最大盤面に到達することができる.
そのため$3\times3$盤面ではプレイヤは$16,352$点を獲得することができる.

さらに$v_{\text{minmax}}$や$v_{\text{maxmax}}$を評価関数として, 通常のルールの環境でプレイさせることを考える.
それぞれminmaxプレイヤ, maxmaxプレイヤと呼ぶことにする.
minmaxプレイヤが通常のルールの環境で獲得する得点の期待値を$v_{\text{minmax-eval}}$とする.
$v_{\text{minmax-eval}}$は, 以下の式~\ref{eq:minmax_eval}に従って後退解析を行い計算できる.
ただし$a_{\text{minmax}}$を, $v_{\text{minmax}}$を評価関数としたときに選択する行動とする.
\begin{align}
    v_{\text{minmax-eval}}(s) =
    \begin{cases}
        0 & (s \text{が終了状態}) \\
        r(s, a_{\text{minmax}}) + \mathbb{E}_{s_\text{next} \in \mathcal{T}(s, a_{\text{minmax}})} v_{\text{minmax-eval}}(s_\text{next}) & (\text{otherwise})
    \end{cases}
    \label{eq:minmax_eval}
\end{align}
maxmaxプレイヤが通常のルールの環境で獲得する得点の期待値$v_{\text{maxmax-eval}}$についても, 全く同じ方法で計算できる.

初期状態の$v_{\text{minmax-eval}}$は$1,363.44$点, $v_{\text{maxmax-eval}}$は
minmaxプレイヤは常に最悪な場合を想定して行動を選択する, 保守的な戦略をとるプレイヤといえる.
ゲームオーバーにつながるような行動を避ける一方で, チャンスも逃し続けてしまうため$v_*$に従って行動を選択する場合に比べて獲得できる得点も大きく減少する.
maxmaxプレイヤは常に楽観的な場合を想定して行動を選択する, ギャンブル的な戦略をとるプレイヤといえる.


\end{document}